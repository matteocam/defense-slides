\section{Conclusions}

\begin{frame}{Summary}
		\begin{itemize}[<+- | alert@+>]	
			\item A critique of the model of rational proofs for multiple delegations;
			%\item A framework for multiple delegations (sequential composability);
			\item (Sequentially composable) protocols for $\log$-depth arithmetic circuits and space-bounded computations.
			\item VC and HE against $\NC^1$ adversaries and computable in low-depth.
			\begin{itemize}
				\item Under minimal assumptions.
			\end{itemize}
		\end{itemize}
\end{frame}

\begin{frame}{Related Work (Results on Rational Proofs)}
\begin{itemize}[<+- | alert@+>]
	\item First rational proof for space-bounded computation
	\item $\cite{ratsumchecks}$ achieves constant-round \textit{rational arguments} for $\P$ (with cryptographic assumptions)
	\begin{itemize}
		\item Our results hold unconditionally.
	\end{itemize}
	\item \cite{rrr16} achieves constant-round interactive proofs for space-bounded computation.
	\begin{itemize}
		\item \cite{rrr16} has better round complexity
		\item We fare better in terms of communication (us: polylog, \cite{rrr16}:$\poly(S(n)n^{\delta})$ and verification (us: sublinear, \cite{rrr16}: $\poly(S((n)) + n$)
	\end{itemize}
\end{itemize}
\end{frame}


\begin{frame}{Related Work (Results on Fine-Grained Protocols)}
	Comparison with information-theoretic results:
	\begin{itemize}[<+- | alert@+>]
		\item ``Proofs for Muggles'' \cite{muggles} obtains constant-round protocols for $\NC^1$.
		\begin{itemize}
			\item results in \cite{muggles} hold unconditionally; 
			\item our verifier runs in $\ACzt$; theirs in $\TC^0$;
			\item our protocol is I/O private and non-interactive.
		\end{itemize}
		\item \cite{gghkr07} obtains constant-depth ($\NC^0$) protocols for $\NC^1$.
		\begin{itemize}
			\item large verifier ($n^3$) and communication overhead.
		\end{itemize}
	\end{itemize}
\end{frame}

\begin{frame}{Future Work}
	\textbf{On Rational Proofs:}
	\begin{itemize}[<+- |  alert@+>]
		\item Rational proofs for $\P$ and $\NP$
		\item validating the model of rational proofs in the real world (How?)
		\item better parameters for the budget for rational proofs 
		% \item right now, the only parameter we use is n and we only require the budget to be poly(n). It'd be great to have a specific budget
		%		 parameter \beta, on the lines of the security parameter in crypto.
		%\item enforcing cost assumptions (with randomized encodings or other approaches)
		%		\item further applications of FG schemes to sequentially composable rational proofs
		%		 \item e.g. can one get a compiler from a stand alone rational scheme to a sequentially composable one having
	\end{itemize}
	\onslide<+->{
		\textbf{On Fine-Grained Crypto:}
		\begin{itemize}[<+- |  alert@+>]
			%\item pushing further on the partial results we have
			\item Other interesting crypto obtainable from $\L \neq \NC^1$ in a fine-grained model?
			\begin{itemize}
				\item Obfuscation? ABE? FE?
			\end{itemize}
			\item All the questions above, but with adversaries of \textit{bounded size} (instead of bounded depth)
			%\item FG primitives in the model of size as first step
		\end{itemize}
	}
\end{frame}
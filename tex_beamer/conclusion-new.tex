\section{Conclusions}

\begin{frame}{Summary}
		\begin{itemize}[<+- | alert@+>]	
			\item A critique of the model of rational proofs for multiple delegations;
			%\item A framework for multiple delegations (sequential composability);
			\item (Sequentially composable) protocols for $\log$-depth arithmetic circuits and space-bounded computations;
			\item VC and HE against $\NC^1$ adversaries and computable in low-depth;
			\begin{itemize}
				\item Under minimal assumptions.
			\end{itemize}
		\end{itemize}
\end{frame}






\begin{frame}{Future Work}
	\textbf{On Rational Proofs:}
	\begin{itemize}[<+- |  alert@+>]
		\item Rational proofs for $\P$ and $\NP$
		\item validating the model of rational proofs in the real world (How?)
		\item better parameters for the budget for rational proofs 
		% \item right now, the only parameter we use is n and we only require the budget to be poly(n). It'd be great to have a specific budget
		%		 parameter \beta, on the lines of the security parameter in crypto.
		%\item enforcing cost assumptions (with randomized encodings or other approaches)
		%		\item further applications of FG schemes to sequentially composable rational proofs
		%		 \item e.g. can one get a compiler from a stand alone rational scheme to a sequentially composable one having
	\end{itemize}
	\onslide<+->{
		\textbf{On Fine-Grained Crypto:}
		\begin{itemize}[<+- |  alert@+>]
			%\item pushing further on the partial results we have
			\item Other interesting crypto obtainable from $\L \neq \NC^1$ in a fine-grained model?
			\begin{itemize}
				\item Obfuscation? ABE? FE?
			\end{itemize}
			\item All the questions above, but with adversaries of \textit{bounded size} (instead of bounded depth)
			%\item FG primitives in the model of size as first step
		\end{itemize}
	}
\end{frame}
\begin{frame}{This Talk}
	\begin{itemize}[<+- | alert@+>]	
	\item A critique of the model of rational proofs for multiple delegations;
	\item A framework for multiple delegations (sequential composability);
	\item Sequentially composable protocols for $\log$-depth arithmetic circuits and space-bounded computations.
	\item Verification schemes secure against $\NC^1$ adversaries. (in progress)
	\end{itemize}
\end{frame}
 % TODO: add open problems, comparison with related work

\begin{frame}{Additional Results}
\begin{itemize}[<+- | alert@+>]
	\item Lower Bounds for Rational Proofs for $\P$ and $\NP$
	\item Composition Theorem for Rational Proofs
	\item Rational Proofs for Randomized Computations
\end{itemize}
\end{frame}

\begin{frame}{Further Applications}
	\begin{itemize}[<+- | alert@+>]
		\item 	Cheap verification of smart contracts
		\begin{itemize}
			\item Something similar done in Ethereum (TrueBit).
		\end{itemize}
		\item Trusted computation with untrusted hardware components.
	\end{itemize}

\end{frame}


\begin{frame}{Rational Model vs Traditional Approaches}
	\begin{block}{What do we gain by assuming rationality?}
		\begin{itemize}[<+- | alert@+>]
			\item Efficiency
			\begin{itemize}
				\item Sublinear verification! (rational proofs)
				\item Low communication (also sublinear). (rational proofs)
			\end{itemize}
			\item Simplicity
			\item Minimal Assumptions
%			\begin{itemize}
%				\item These results hold even if OWFs do not exist
%				\item (\textit{Out of scope:} approaches using rationality+crypto )
%			\end{itemize}
		\end{itemize}
	\end{block}
	
%	\pause
%	% - The gains: efficiency, simplicity, lack of crypto
%	\begin{block}{Contrast with information-theoretic schemes:}
%		\begin{itemize}[<+- | alert@+>]
%			\item Comparable communication and worker's overhead (e.g [GKR08] vs [GHR16,\textbf{C}G15]; [RRR16] vs [\textbf{C}G17])
%			\item Verification is at least linear in information-theoretic setting.
%		\end{itemize}
%	\end{block}
\end{frame}




\begin{frame}{Open Problems}
	\textbf{On Rational Proofs:}
	\begin{itemize}[<+- |  alert@+>]
		\item Rational proofs for $\P$ and $\NP$
		\item validating the model of rational proofs in the real world (How?)
		\item better parameters for the budget for rational proofs 
		% \item right now, the only parameter we use is n and we only require the budget to be poly(n). It'd be great to have a specific budget
		%		 parameter \beta, on the lines of the security parameter in crypto.
		\item enforcing cost assumptions (with randomized encodings or other approaches)
		%		\item further applications of FG schemes to sequentially composable rational proofs
		%		 \item e.g. can one get a compiler from a stand alone rational scheme to a sequentially composable one having
	\end{itemize}
	\onslide<+->{
		\textbf{On crypto against limited adversaries:}
		\begin{itemize}[<+- |  alert@+>]
			%\item pushing further on the partial results we have
			\item Other interesting crypto obtainable from $\L \neq \NC^1$ in a fine-grained model?
			\begin{itemize}
				\item Obfuscation? ABE? FE?
			\end{itemize}
			\item All the questions above, but with adversaries of \textit{bounded size} (instead of bounded depth)
			%\item FG primitives in the model of size as first step
		\end{itemize}
	}
	
\end{frame}

\begin{frame}{}
	\Large Fin.
\end{frame}
